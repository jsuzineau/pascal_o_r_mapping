@@GettingStartedE.html
<TITLE Virtual Treeview step by step>


<PRE>
Written by Sven H. (<EXTLINK mailto:h.sven@gmx.at>h.sven@gmx.at</EXTLINK>), Revision and translation by Mike Lischke (<EXTLINK mailto:public@delphi-gems.com>public@delphi-gems.com</EXTLINK>)
</PRE>


At the time when this description was created I had not much Delphi knowledge and had not yet read through any of my two
Delphi books. But I was quite impatient and wanted to try out what is possible. Although I have some knowledge about
\object oriented programming and C++ (I have learned something about it during my studies), this project was my first
attempt to program in Delphi. It could be that I have not provided the most elegant solutions und I am always open for
improvement suggestions. But all principles I demonstrated here do work (at least for me J). I have implemented them in
my first project this way. This guidance is made in the first place for programmers who are not yet familiar with Virtual
Treeview and will so perhaps have an easier start. If you have questions or suggestions regarding this guidance please
forward them to <EXTLINK mailto:h.sven@gmx.at>h.sven@gmx.at</EXTLINK>. For other questions you can contact Mike and use
the dedicated newsgroup, respectively.



I am neither a Virtual Treeview nor a Delphi expert and have collected all the answers (with the help of Mike) with quite
some effort. In order to avoid the afterwards relatively simple things to become problematic I have written this short
guidance. The real problems will appear later.

� 2001 The parts in this guidance beyond the text from the online help are copyrighted. Every publication requires my
admission.

Have fun with it, Sven.



* Preparations *
Before we start some preparations are necessary:

    * Place a Virtual Treeview component on a form.
    * Change the properties as you like.
    * A record for node data must be defined.



In order to store the own node data some musing is important. How shall the record look like?

<B>a) All nodes in the tree are equal</B>

In this case a simple record defines the necessary data structure, e.g.:


<CODE>
<B>type</B>
  rTreeData = <B>record</B>
  Text: WideString;
  URL: <B>string</B>[<COLOR Pink>255</COLOR>];
  CRC: LongInt;
  isOpened: Boolean;
  ImageIndex: Integer;
<B>end</B>;
</CODE>


<B>b) There are different nodes in the tree (e.g. folders that can have sub nodes)</B>

I will follow this case because my tree will hold folders, which can in turn get own nodes. Since I intent to store
created trees in a file in order to restore them later further deliberations are necessary: Suppose a folder node has
\only a name and a leaf node has a name and a text info field. Potentially, I also want to store a second kind of leaf
node, which will for instance have a number instead of the text field. The problem in the context of reading data form a
stream is that I must know which data is stored in which order in the stream, because I have to read it in exactly the
same order again. Hence I have to determine from the very first information in the stream what information will follow.
For instance there is a node name, but then? Is there nothing more or another text information (string) or even an
integer value? I think the point is clear. The first data, which I read, has to carry this information.



These deliberations have leaded me to the following solution: I save now in the stream [label]-\>[name]-\>[following
data]



0 -\> 'Folder'

1 -\> 'Info node' -\> 'Blabla'

2 -\> 'Number node' -\> 123



I know from the stream I always read an integer value first. Depending whether this is 0, 1 or 2 I have to read - now
known - following values. Now let us consider the record.


<CODE>
<B>type</B>
  rTreeData = <B>record</B>
  Typ: Integer;
  Name: <B>string</B>[<COLOR Pink>255</COLOR>];
  pNodeData: Pointer;
<B>end</B>;
</CODE>


Hey, there is suddenly a pointer in the record. Well, here are some additional comments:

      1. Typ is an integer value, from which I can determine what kind of node this is, in my example 1, 2 or 3.
      2. Name is the name of the node. This will be needed relatively often because it is also seen as part of the tree
         and I want to access this information easily (man, I am lazy).
      3. The pointer allows (similar to the data property of the tree) a record or even better a class instance to
         connect.



Now I still have the freedom to define a base class of node. It contains all properties and methods, which all classes
will share. And from this I can derive proper sub classes (e.g. text nodes, value nodes etc.). An additional advantage of
this record is its fixed size. Hence you can always return the same size in case the tree asks for it (see also property
NodeDataSize), but more about that later.

Just one remark: If you don't want to use classes you can also simply define 3 records, which define as first element, a
type and which react differently depending on this type.



<B><U>Alternative solution:</U></B>

Okay, I admit it. It would of course also be possible to write the type into the stream and read it from the stream
separately without saving it as part of the record. The type of the node class is indirectly known because you can ask a
class which class name it has (see e.g. class function ClassName) and the class knows it too. So I shall store a node,
\okay. I pass on the stream to the Node.SaveToFile(Stream) method, which writes, depending on which node class we
actually have, automatically the value 1, 2 or 3 into the stream.



During load from stream I read first the value 1, 2 or 3 and decide what class is meant. Then I create an instance of
this class and call its LoadFromFile method. Well, this solution is my most preferred and before another one enters my
brain I will implement it (Note: in step 5 I will change something).



<U>So I do following:</U>



As you can see from the declaration of the internal node of Virtual Tree


<CODE>
TVirtualNode = <B>packed</B> <B>record</B>
  Index, <COLOR Blue>// index of node with regard to its parent</COLOR>
  ChildCount: Cardinal; <COLOR Blue>// number of child nodes</COLOR>
  ...
  ...
  LastChild: PVirtualNode; <COLOR Blue>// link to the node's last child...</COLOR>
  Data: <B>record</B> <B>end</B>; <COLOR Blue>// this is a placeholder, each node gets extra</COLOR>
  <COLOR Blue>// data determined by NodeDataSize</COLOR>
<B>end</B>;
</CODE>


there is another record at the end of the record structure. Which exact structure this is will be determined indirectly.


<CODE>
<B>type</B>
  rTreeData = <B>record</B>
    Name: <B>string</B>[<COLOR Pink>255</COLOR>]; <COLOR Blue>// the identifier of the node</COLOR>
    ImageIndex: Integer; <COLOR Blue>// the image index of the node</COLOR>
    pNodeData: Pointer;
<B>  end</B>;
</CODE>


Let the above record be the structure. The Virtual Treeview does not really know this structure, but it knows how much
space must be reserved. We tell it by


<CODE>
myVirtualTree.NodeDataSize := SizeOf(rTreeData);
</CODE>


\Note, even if you want to store only one value, e.g. a pointer as node data, simply return the size, which should be
reserved.



<B>Implementation</B>

<I>An empty tree</I>



I begin with an empty tree (no top level nodes are created at design time):

    * Either an existing tree is read from a file or
    * A top-level node is created.

*  *
Before a node can be created you have to determine the size of the actual node data. According to the docs there are
three opportunities:

    * In the object inspector
    * In the OnGetNodeDataSize - event or
    * During creation of the form



I decide to use the last variant and will now do the following during form creation:


<CODE>
<B>procedure</B> TMyForm.FormCreate(Sender: TObject);

<B>var</B>
  Node: PVirtualNode;

<B>begin</B>
  ...
  <COLOR Blue>// create tree</COLOR>
  MyTree.NodeDataSize := SizeOf(TTreeData);
  <B>if</B> MyForm.filename = <COLOR Pink>''</COLOR> <B>then</B> <B>begin</B> <COLOR Blue>// if there is no tree to load</COLOR>
    <COLOR Blue>// create tree with top level node</COLOR>
    Node := BookmarkForm.BookmarkTree.AddChild(nil); <COLOR Blue>// adds a top level node</COLOR>
<B>  end
  else
  begin</B>
    <COLOR Blue>// load tree</COLOR>
    ....
<B>  end</B>;
  ....
<B>end</B>;
</CODE>


<B><I>Data for the node</I></B>



After the call of AddChild data can be assigned. For this a pointer to the self-defined record will be declared and via
the function GetNodeData connected with the correct address. By using this pointer we can now access the elements of the
record and assign them values.


<CODE>
<B>var</B>
  ...
  NodeData: ^rTreeData;

<B>begin</B>
  ...
  <COLOR Blue>// determine data for node</COLOR>
  NodeData := BookmarkForm.BookmarkTree.GetNodeData(Node);
  NodeData.Name := <COLOR Pink>'new project'</COLOR>;
  NodeData.ImageIndex := <COLOR Pink>0</COLOR>;
  ...
</CODE>


<B><I>Show the node name</I></B>



The name of the node shall now appear as node identification in the tree. All data about the node as well as the name are
unknown to the treeview and it has to query for them.



Every time the identification of the node is needed an event OnGetText will be triggered. In the event handler we return
the name of the node in the variable Text. Nothing more is needed.


<CODE>
<B>procedure</B> TBookmarkForm.BookmarkTreeGetText(Sender: TBaseVirtualTree;
  Node: PVirtualNode; Column: Integer; TextType: TVSTTextType; <B>var</B> Text: WideString);
<B>
var</B>
  NodeData: ^rTreeData;

<B>begin</B>
  NodeData := Sender.GetNodeData(Node);
  <COLOR Blue>// return identifier of the node</COLOR>
  Text := NodeData.Name;
<B>end</B>;
</CODE>


<B><I>The icon for the node</I></B>



Because I like it colorful I want also to provide an icon for the top-level node. Following steps are necessary to
accomplish that:



    * A TImageList must be placed onto the form and filled with images
    * The property Images of the VirtualTreeview gets assigned this image list
    * Implement an OnGetImageIndex event handler.



In the event OnGetImageIndex you can the index be determine which determines in turn which image form the list must be
shown.



Because the method is also called for the state icons but I do not want yet to state icons (but I already have assigned
and image list to the property StateImages) the value for this case (Kind � ikState) is -1.


<CODE>
<B>procedure</B> TBookmarkForm.BookmarkTreeGetImageIndex(Sender: TBaseVirtualTree;
  Node: PVirtualNode; Kind: TVTImageKind; Column: Integer; <B>var</B> Index: Integer);

<B>var</B>
  NodeData: ^rTreeData;

<B>begin</B>
  NodeData := Sender.GetNodeData(Node);
<B>  case</B> Kind <B>of</B>
    ikState: <COLOR Blue>// for the case the state icon has been requested</COLOR>
      Index := <COLOR Pink>-1</COLOR>;
    ikNormal, ikSelected: <COLOR Blue>// normal or the selected icon is required</COLOR>
      Index := NodeData.ImageIndex;
<B>  end</B>;
<B>end</B>;
</CODE>


Depending on whether a node is selected or not, different icons shall be shown (see step 6).

<B><I>

Only one node class in the record</I></B>



Since I want to avoid mixing data in the record and later then data in the node class I decided to change this record


<CODE>
<B>type</B>
  TTreeData = <B>record</B>
    Name: <B>string</B>[<COLOR Pink>255</COLOR>]; <COLOR Blue>// the identifier of the node</COLOR>
    ImageIndex: Integer; <COLOR Blue>// the image index of the node</COLOR>
    pNodeData: Pointer;
<B>end</B>;
</CODE>


into a record which contains only one pointer to a node class. I declare therefore first a node class


<CODE>
TBasicNodeData = <B>class</B>
  ...
<B>end</B>;
</CODE>


and then a structure of the form:


<CODE>
rTreeData = <B>record</B>
  BasicND: TBasicNodeData;
<B>end</B>;
</CODE>


This record always needs 4 bytes for the pointer to the class.



Particular attention is to direct to the event OnGetText. This event will already be called during creation of the node
with Tree.AddChild(nil) in order to determine the space the new node's caption will need (but only if no columns were
created). At this point however the node class could not yet be initialised (no constructor call yet). Hence for this
case


<CODE>
<B>if</B> NodeD.BasicND = <B>nil</B> <B>then</B>
  Text := <COLOR Pink>''</COLOR>
</CODE>


must be returned or you wrap the entire initialization into a BeginUpdate/EndUpdate block and initialized the nodes
before EndUpdate is called (e.g. by ValidateNode(Node)).*



Without this provision an access violation would be the result.



*  *
* Example class declaration *

<CODE><B>unit</B> TreeData;

<B>interface</B>

<COLOR Blue>//===========================================</COLOR>

<B>type</B>
  <COLOR Blue>// declare common node class</COLOR>
  TBasicNodeData = <B>class</B>
  <B>protected</B>
    cName: ShortString;
    cImageIndex: Integer;
  <B>public</B>
    <B>constructor</B> Create; <B>overload</B>;
    <B>constructor</B> Create(vName: ShortString; vIIndex: Integer = 0); <B>overload</B>;

    <B>property</B> Name: ShortString read cName write cName;
    <B>property</B> ImageIndex: Integer read cImageIndex write cImageIndex;
  <B>end</B>;

  <COLOR Blue>// declare new structure for node data</COLOR>
  rTreeData = <B>record</B>
    BasicND: TBasicNodeData;
  <B>end</B>;

<B>implementation</B>

<B>constructor</B> TBasicNodeData.Create;
<B>begin</B>
   <COLOR Blue>{ not necessary
   cName := '';
   cImageIndex := 0;
   }</COLOR>
<B>end</B>;

<B>constructor</B> TBasicNodeData.Create(vName: ShortString; vIIndex: Integer = 0);
<B>begin</B>
  cName := vName;
  cImageIndex := vIIndex;
<B>end</B>;

<B>end</B>.
</CODE>


* Example creation of the tree *

<CODE><COLOR Blue>// Tree will be created when the form is created.</COLOR>
<B>procedure</B> TMyForm.FormCreate(Sender: TObject);

<B>var</B>
   Node: PVirtualNode;
   NodeD: ^rTreeData;

<B>begin</B>
  ....
  <COLOR Blue>// create tree</COLOR>
  MyTree.NodeDataSize := SizeOf(rTreeData);
  <B>if</B> MainControlForm.filename = <COLOR Pink>''</COLOR> <B>then</B>
  <B>begin</B>
    <COLOR Blue>// create tree with top level node</COLOR>
    Node := MyTree.AddChild(<B>nil</B>); <COLOR Blue>// adds a node to the root of the tree</COLOR>
    <COLOR Blue>// assign data for this node</COLOR>
    NodeD := MyTree.GetNodeData(Node);
    NodeD.BasicND := TBasicNodeData.Create(<COLOR Pink>'new project'</COLOR>);
  <B>end</B>
  <B>else</B>
  <B>begin</B>
    <COLOR Blue>// load tree</COLOR>
  <B>end</B>;
  ...
<B>end</B>;

<COLOR Blue>// returns the text (the identification) of the node</COLOR>
<B>procedure</B> TMyForm.MyTreeGetText(Sender: TBaseVirtualTree; Node: PVirtualNode; Column: Integer;
  TextType: TVSTTextType; <B>var</B> Text: WideString);

<B>var</B>
   NodeD: ^rTreeData;

<B>begin</B>
  NodeD := Sender.GetNodeData(Node);

  <COLOR Blue>// return the identifier of the node</COLOR>
  <B>if</B> NodeD.BasicND = <B>nil</B> <B>then</B>
    Text := <COLOR Pink>''</COLOR>
  <B>else</B>
    Text := NodeD.BasicND.Name;
<B>end</B>;

<COLOR Blue>// returns the index for image display</COLOR>
<B>procedure</B> TMyForm.MyTreeGetImageIndex(Sender: TBaseVirtualTree;
   Node: PVirtualNode; Kind: TVTImageKind; Column: Integer; <B>var</B> Index: Integer);

<B>var</B>
   NodeD: ^rTreeData;

<B>begin</B>
  NodeD := Sender.GetNodeData(Node);

  <B>case</B> Kind <B>of</B>
    ikState: <COLOR Blue>// for the case the state index has been requested</COLOR>
      Index := <COLOR Pink>-1</COLOR>;
    ikNormal, ikSelected: <COLOR Blue>// normal icon case</COLOR>
      Index := NodeD.BasicND.ImageIndex;
   <B>end</B>;
<B>end</B>;
</CODE>


* Icons for selected nodes *
If a node is selected a different symbol shall be shown. Therefore I implement a new method


<CODE>
  <B>function</B> GetImageIndex(focus: Boolean): Integer; <B>virtual</B>;
</CODE>


which gets the normal image index or the index for focused nodes depending on whether the node has the focus or not.



Call:
<CODE>
   Index := NodeD.BasicND.GetImageIndex(Node = Sender.FocusedNode);
</CODE>


Implementation of the method:


<CODE>
<B>function</B> TBasicNodeData.GetImageIndex(focus: Boolean): Integer;

<B>begin</B>
  <B>if</B> focus <B>then</B>
    \Result := cImageIndexFocus
  <B>else</B>
    \Result := cImageIndex;
<B>end</B>;
</CODE>


where cImageIndex has always the normal index and cImageIndex Focus the index for focused nodes. I assume in this case
that the selected index is always one more than the normal index. To ensure this, the constructor is changed this way:


<CODE>
<B>constructor</B> TBasicNodeData.Create(vName: ShortString; vIIndex: Integer = 0);

<B>begin</B>
  cName := vName;
  cImageIndex := vIIndex;
  cImageIndexFocus := vIIndex + <COLOR Pink>1</COLOR>;
<B>end</B>;
</CODE>


* Adding and deleting nodes *
In order to implement and test more functions I want finally an opportunity to create the tree. By using a context menu
is shall be possible to add and remove nodes.



Hence I define a popup menu with two entries: [Add] and [Remove]. To have the clicked node getting the focus the option
voRightClickSelect must be set to True.



So if Add has been chosen a child node will be created for the focused node:


<CODE>
<B>procedure</B> TMyForm.addClick (Sender: TObject);

<B>var</B>
  Node: PVirtualNode;
  NodeD: ^rTreeData;

<B>begin</B>
  <COLOR Blue>// Ok, a node must be added.</COLOR>
  Node := MyTree.AddChild(MyTree.FocusedNode); <COLOR Blue>// adds a node as the last child</COLOR>
  <COLOR Blue>// determine data of node</COLOR>
  NodeD := MyTree.GetNodeData(Node);
  NodeD.BasicND := TBasicNodeData.Create(<COLOR Pink>'Child'</COLOR>);
<B>end</B>;
</CODE>


Caution: What must be done if no node has the focus?

\-\> e.g. insert the new node as child of a top level nodes.


<CODE>
  <B>if</B> BookmarkTree.FocusedNode = <B>nil</B> <B>then</B>
  <B>begin</B>
    <COLOR Blue>// insert as child of the first top level node</COLOR>
    Node := BookmarkTree.AddChild(BookmarkTree.RootNode.FirstChild);
    <COLOR Blue>// determine data for node</COLOR>
    NodeD := BookmarkTree.GetNodeData(Node);
    NodeD.BasicND := TFolderNodeData.Create(<COLOR Pink>'new folder'</COLOR>);
  <B>end
  else
  begin</B>
    <COLOR Blue>// Ok, a new node must be added.</COLOR>
    Node := BookmarkTree.AddChild(BookmarkTree.FocusedNode);
    <COLOR Blue>// determine data of the node</COLOR>
    NodeD := BookmarkTree.GetNodeData(Node);
    NodeD.BasicND := TFolderNodeData.Create(<COLOR Pink>'new folder'</COLOR>);
  <B>end</B>;
</CODE>




If the node with the focus must be deleted the following happens:


<CODE>
<B>procedure</B> TMyForm.delClick (Sender: TObject);

<B>begin</B>
  <COLOR Blue>// The focused node should be removed. The top level must not be deleted however.</COLOR>
  <B>if</B> MyTree.FocusedNode = <B>nil</B> <B>then</B>
    MessageDlg(<COLOR Pink>'There was no node selected.'</COLOR>, mtInformation, [mbOk], 0)
  <B>else</B>
    <COLOR Blue>// Note: RootNode is the internal (hidden) root node and parent of all top
    // level nodes. To determine whether a node is a top level node you also use
    // GetNodeLevel which returns 0 for top level nodes.</COLOR>
  <B>if</B> MyTree.FocusedNode.Parent = MyTree.RootNode <B>then</B>
    MessageDlg(<COLOR Pink>'The project node must not be deleted.'</COLOR>, mtInformation, [mbOk], 0)
  <B>else</B>
    MyTree.DeleteNode(MyTree.FocusedNode);
<B>end</B>;
</CODE>


I want to prevent, however, that the top-level node gets deleted. Hence I check with the comparison
MyTree.FocusedNode.Parent = MyTree.RootNode whether the focused node is not a top-level node. Here you have to consider
that the property RootNode returns the (hidden) internal root node, which is the common parent of all top-level nodes.



While we are at deleting nodes:

Every data of the record is automatically free as soon as this is required. In this case it is not enough, however, to
free the memory, which holds the pointer to the class (object instance), but it is also necessary to free the memory,
which is allocated by the class itself. This happens by calling the destructor of the class in the OnFreeNode event:


<CODE>
<B>procedure</B> TMyForm.MyTreeFreeNode(Sender: TBaseVirtualTree; Node: PVirtualNode);

<B>begin</B>
  <COLOR Blue>// Free here the node data (Note: type PtreeData = ^rTreeData).</COLOR>
  PTreeData(Sender.GetNodeData(Node)).BasicND.Free;
<B>end</B>;
</CODE>


* Adding folder and leafs *
Now I am ready to add folders to the tree as well as final nodes, which do not have children. For this I derive two new
node classes from the base class.


<CODE>
TFolderNodeData = <B>class</B>(TBasicNodeData)
TItemNodeData = <B>class</B>(TBasicNodeData)
</CODE>


Depending on which kind of node the user wants to create using the context menu I store a particular class in the node
record.


<CODE>
NodeD.BasicND := TFolderNodeData.Create(<COLOR Pink>'new folder'</COLOR>);
NodeD.BasicND := TItemNodeData.Create(<COLOR Pink>'new node'</COLOR>);
</CODE>


These classes contain a new property ChildrenAllowed. Based on this property you can now distinct whether the node with
the focus may get children (folder) or not (items).



* Storing the tree *
Now I can finally implement storing the tree. I have already thought a lot about this step. Let us see if this was
worthwhile.



Again a quote from Preparations:

I want to store a node, okay. I hand over the stream to the MyNodeClass.SaveToFile method and this method writes
depending upon which node class it actually is automatically the value 1, 2 or 3 as a kind of class ID into the stream
(alternatively you can use an enumeration type).



During load I read first the value 1, 2 or 3 from the stream and decide based on it which class we deal with. Then I
create an instance of this class and call its method LoadFromFile.



Hint:

It would also be possible to store the class name instead of the ID for the class. During read and creation of the class
\one could use class references and virtual constructors and save so the case-statement as I did in the OnLoadNode event,
to decide which class instance must be created (example see Delphi 5, written by Elmar Warken, Addison-Wesley, chapter
4.3.3, page 439).



Before you can read something it must be written first. Hence I will first implement the necessary procedures to store
the tree. Since we care ourselves that the identification of the node gets saved the option toSaveCaption can be removed
from StringOptions. This way data is not stored twice.



For saving the tree the procedure


<CODE>
<B>procedure</B> TBaseVirtualTree.SaveToFile(<B>const</B> FileName: TFileName);
</CODE>


is called. Thereby the structure of the tree is automatically stored. In order to save our additional data there is an
event OnSaveNode where we can simply store our data into the provided stream.


<CODE>
<B>property</B> OnSaveNode: TVTSaveNodeEvent <B>read</B> FOnSaveNode <B>write</B> FOnSaveNode;
</CODE>


If OnSaveNode is triggered then the method SaveNode of the particular node class will be called:


<CODE>
<B>procedure</B> TMyForm.MyTreeSaveNode(Sender: TBaseVirtualTree; Node: PVirtualNode; Stream: TStream);

<B>begin</B>
  PTreeData(Sender.GetNodeData(Node)).BasicND.SaveToFile(Stream);
<B>end</B>;
</CODE>


In the SaveNode method of the class fields like node name, image index etc. are stored in the tree:


<CODE>
<B>procedure</B> TBasicNodeData.SaveNode(Stream: TStream);

<B>var</B>
  size: Integer;

<B>begin</B>
  <COLOR Blue>// save type of the node</COLOR>
  Stream.Write(Art, SizeOf(Art));

  <COLOR Blue>// store cName</COLOR>
  Size := Length(cName) + <COLOR Pink>1</COLOR>; <COLOR Blue>// include terminating #0</COLOR>
  Stream.Write(Size, SizeOf(Size)); <COLOR Blue>// store length of the string</COLOR>
  Stream.Write(PChar(cName)^, Size); <COLOR Blue>// now the string itself</COLOR>

  <COLOR Blue>// store cImageIndex</COLOR>
  Stream.Write(cImageIndex, SizeOf(cImageIndex));

  <COLOR Blue>// store cImageIndexFocus</COLOR>
  Stream.Write(cImageIndexFocus, SizeOf(cImageIndexFocus));

  <COLOR Blue>// store cChildrenAllowed</COLOR>
  Stream.Write(cChildrenAllowed, SizeOf(cChildrenAllowed));
<B>end</B>;

Now we can the tree we save also load again. This process could look like:

<B>try</B>
  <COLOR Blue>// load tree</COLOR>
  MyTree.LoadFromFile(MainControlForm.Filename);
<B>except</B>
  <B>on</B> E: Exception <B>do</B>
  <B>begin</B>
    Application.MessageBox(PChar(E.Message), PChar(<COLOR Pink>'Error while loading.'</COLOR>), MB_OK);
    MainControlForm.Filename := <COLOR Pink>''</COLOR>;

    <COLOR Blue>// create tree with top level node (since loading failed)</COLOR>
    Node := MyTree.AddChild(nil);
    NodeD := MyTree.GetNodeData(Node);
    NodeD.BasicND := TBasicNodeData.Create(<COLOR Pink>'new project'</COLOR>);
  <B>end</B>;
<B>end</B>;
</CODE>


By the call of LoadFromFile the event OnLoadNode will be triggered and consequently the method LoadNode:


<CODE>
<B>procedure</B> TBasicNodeData.LoadNode(Stream: TStream);

<B>var</B>
  Size: Integer;
  StrBuffer: PChar;

<B>begin</B>
  <COLOR Blue>// load cName</COLOR>
  Stream.Read(Size, SizeOf(Size)); <COLOR Blue>// length of the string</COLOR>

  StrBuffer := AllocMem(Size); <COLOR Blue>// get temporary memory</COLOR>
  Stream.Read(StrBuffer^, Size); <COLOR Blue>// read the string</COLOR>
  cName := StrBuffer;
  FreeMem(StrBuffer);
  <COLOR Blue>// Alternatively you can simply use:
  // SetLength(cName, Size);
  // Stream.Read(PChar(cName)^, Size);</COLOR>

  <COLOR Blue>// load cImageIndex</COLOR>
  Stream.Read(cImageIndex, SizeOf(cImageIndex));

  <COLOR Blue>// load cImageIndexFocus</COLOR>
  Stream.Read(cImageIndexFocus, SizeOf(cImageIndexFocus));

  <COLOR Blue>// load cChildrenAllowed</COLOR>
  Stream.Read(cChildrenAllowed, SizeOf(cChildrenAllowed));
<B>end</B>;
</CODE>


* Two columns in the treeview *
Now I want to show two columns in the treeview. Therefore I set the new properties of the tree in the object inspector.



By using Header.Columns you can create the desired columns. After that, you only have to set Header.Options.hoVisible to
True and the columns will appear in the treeview.



After you have set all necessary options you can give now the text and the icon for the particular column, respectively.
This happens in the already existing event handlers OnGetText and OnGetImageIndex where now also the given column index
must be taken into account.


<CODE>
<B>procedure</B> TMyForm.MyTreeGetText(Sender: TBaseVirtualTree; Node: PVirtualNode;
  Column: Integer; TextType: TVSTTextType; <B>var</B> Text: WideString);

<B>var</B>
  NodeD: ^rTreeData;

<B>begin</B>
  NodeD := Sender.GetNodeData(Node);

  <COLOR Blue>// return the the identifier of the node</COLOR>
  <B>if</B> NodeD.BasicND = <B>nil</B> <B>then</B>
    Text := ''
  <B>else</B>
  <B>begin</B>
    <B>case</B> Column <B>of</B>
      \-1,
      0: <COLOR Blue>// main column, -1 if columns are hidden, 0 if they are shown</COLOR>
        Text := NodeD.BasicND.Name;
      1:
        Text := <COLOR Pink>'This text appears in column 2.'</COLOR>;
    <B>end</B>;
  <B>end</B>;
<B>end</B>;

<B>procedure</B> TMyForm.MyTreeGetImageIndex(Sender: TBaseVirtualTree; Node: PVirtualNode;
  Kind: TVTImageKind; Column: Integer; <B>var</B> Index: Integer);

<B>var</B>
  NodeD: ^rTreeData;

<B>begin</B>
  NodeD := Sender.GetNodeData(Node);

  <B>if</B> Column = 0 <B>then</B> <COLOR Blue>// icons only in the first column</COLOR>
    <B>case</B> Kind <B>of</B>
      ikState:
        Index := <COLOR Pink>-1</COLOR>;
      ikNormal, ikSelected:
        Index := NodeD.BasicND.GetImageIndex(Node = Sender.FocusedNode);
      ikOverlay: <COLOR Blue>// e.g. to mark a node whose content changed,</COLOR>
        <COLOR Blue>// Note: don't forget to call ImageList.Overlay for the image.</COLOR>
        <B>if</B> NodeD.BasicND.ImageIndex = 4 <B>then</B>
          Index := 6;
    <B>end</B>;
<B>end</B>;
</CODE>


* Accessing the columns *
I want to demonstrate the access to the columns of a TVirtualStringTrees based on an example. In order to store global
\options, as in Point 2.12 I want to know the width of a column. This information is updated every time an OnColumnResize
event is triggered:


<CODE>
<B>procedure</B> TBookmarkForm.BookmarkTreeColumnResize(Sender: TBaseVirtualTree; Column: Integer);

<B>var</B>
  NodeD: PTreeData;

<B>begin</B>
  NodeD := Sender.GetNodeData(Sender.RootNode.FirstChild);

  <COLOR Blue>// Keep the new size of the column in the project node.</COLOR>
  TProjectNodeData(NodeD.BasicND).SetHColumnsWidth(
    TVirtualStringTree(Sender).Header.Columns.Items[Column].Width,Column);
<B>end</B>;
</CODE>


The exciting part is the type casting of the sender object. In TBaseVirtualTree the header property is protected and only
after conversion (casting) to TVirtualTree it becomes accessible.



* Global tree options *
Global options like the sizes of the columns, which are adjusted in the project, will be stored as properties of the
top-level node. It contains so all project related options.



In order to avoid that all derived classes inherit these fields the top-level node class will be build from a new project
node class, which will be derived from the base node class.



The new hierarchy looks now so:

� Base node class... unites the properties of all nodes

� Project node class... enriches the base with management of project related options

� Folder node classes... enriches the base with default properties for all leaf nodes

� Leaf node class... the actual node class (special properties)



Since this involves already very application specific program details I want only make some notes.



The base node class has the ability to store node data. These methods must be declared as virtual and will be overridden
in the project node class to allow saving the project data.



Well, now I am ready to work with VirtualTreeview. It will become interesting later again when I will try to drag data
from other applications to the tree. But this is a different story...

Summary
Often a simple step by step tutorial gets you much faster started than a long list of features and possibilities. This
topic describes the basic usage on the basis of a simple project.
